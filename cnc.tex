%===================================================================================
% JORNADA CIENTÍFICA ESTUDIANTIL - MATCOM, UH
%===================================================================================
% Esta plantilla ha sido diseñada para ser usada en los artículos de la
% Jornada Científica Estudiantil de MatCom.
%
% Por favor, siga las instrucciones de esta plantilla y rellene en las secciones
% correspondientes.
%
% NOTA: Necesitará el archivo 'jcematcom.sty' en la misma carpeta donde esté este
%       archivo para poder utilizar esta plantila.
%===================================================================================



%===================================================================================
% PREÁMBULO
%-----------------------------------------------------------------------------------
\documentclass[a4paper,10pt,twocolumn]{article}

%===================================================================================
% Paquetes
%-----------------------------------------------------------------------------------
\usepackage{amsmath}
\usepackage{amsfonts}
\usepackage{amssymb}
\usepackage{jcematcom}
\usepackage[utf8]{inputenc}
\usepackage{listings}
\usepackage[pdftex]{hyperref}
\usepackage{caption}
\usepackage{subcaption}
%-----------------------------------------------------------------------------------
% Configuración
%-----------------------------------------------------------------------------------
\hypersetup{colorlinks,%
	    citecolor=black,%
	    filecolor=black,%
	    linkcolor=black,%
	    urlcolor=blue}

%===================================================================================



%===================================================================================
% Presentacion
%-----------------------------------------------------------------------------------
% Título
%-----------------------------------------------------------------------------------
\title{Una estrategia de Meta-Learning para flujos genéricos de AutoML}

%-----------------------------------------------------------------------------------
% Autores
%-----------------------------------------------------------------------------------
\author{\\
\name Loraine Monteagudo García \email \href{loraine.monteagudo@matcom.uh.cu}{loraine.monteagudo@matcom.uh.cu}
	% \\ \addr Grupo B612 
	}

%-----------------------------------------------------------------------------------
% Tutores
%-----------------------------------------------------------------------------------
\tutors{\\
Dr. Suilan Estévez Velarde, \emph{Universidad de La Habana} \\
Lic. Daniel Alejandro Valdés Pérez, \emph{Universidad de La Habana}}

%-----------------------------------------------------------------------------------
% Headings
%-----------------------------------------------------------------------------------
\jcematcomheading{\the\year}{1-\pageref{end}}{L. Monteagudo}

%-----------------------------------------------------------------------------------
\ShortHeadings{Una estrategia de Meta-Learning}{L. Monteagudo}
%===================================================================================



%===================================================================================
% DOCUMENTO
%-----------------------------------------------------------------------------------
\begin{document}

%-----------------------------------------------------------------------------------
% NO BORRAR ESTA LINEA!
%-----------------------------------------------------------------------------------
\twocolumn[
%-----------------------------------------------------------------------------------

\maketitle

%===================================================================================
% Resumen y Abstract
%-----------------------------------------------------------------------------------
\selectlanguage{spanish} % Para producir el documento en Español

%-----------------------------------------------------------------------------------
% Resumen en Español
%-----------------------------------------------------------------------------------
\begin{abstract}

	El campo de aprendizaje de máquinas automático (AutoML) se ha destacado como una de las principales alternativas para encontrar buenas soluciones para problemas complejos de aprendizaje automático. A pesar del reciente éxito de AutoML, todavía quedan muchos desafíos. El aprendizaje de AutoML es un proceso costoso en tiempo y puede llegar a ser ineficiente computacionalmente. Meta-Learning es descrito como el proceso de aprender de experiencias pasadas aplicando varios algoritmos de aprendizaje en diferentes tipos de datos y, por lo tanto, reduce el tiempo necesario para aprender nuevas tareas. Una de las ventajas de las técnicas de meta-learning es que pueden servir como un apoyo eficiente para el proceso de AutoML, aprendiendo de tareas previas los mejores algoritmos para resolver un determinado tipo de problema. De esta manera, es posible acelerar el proceso de AutoML, obteniendo mejores resultados en el mismo período de tiempo. El objetivo de esta tesis es diseñar una estrategia de meta-learning para dominios genéricos en el aprendizaje automático.

\end{abstract}

%-----------------------------------------------------------------------------------
% English Abstract
%-----------------------------------------------------------------------------------
\vspace{0.5cm}

\begin{enabstract}
	The field of automated machine learning (AutoML) has been highlighted as one of the main alternatives for finding good solutions for complex machine learning problems. Despite the recent success of AutoML, many challenges remain. Learning AutoML is a time-consuming process and can be computationally inefficient. Meta-learning is described as the process of learning from past experiences by applying various learning algorithms on different types of data, and therefore reduces the time required to learn new tasks. One of the advantages of meta-learning techniques is that they can serve as an efficient support for the AutoML process, learning from previous tasks the best algorithms to solve a certain type of problem. In this way, it is possible to speed up the AutoML process, obtaining better results in the same period of time. The objective of this thesis is to design a meta-learning strategy for generic domains in machine learning.

\end{enabstract}

%-----------------------------------------------------------------------------------
% Palabras clave
%-----------------------------------------------------------------------------------
\begin{keywords}
	Aprendizaje Automático, Meta-Learning, AutoML
\end{keywords}

%-----------------------------------------------------------------------------------
% Temas
%-----------------------------------------------------------------------------------
\begin{topics}
	Inteligencia Artificial, Aprendizaje Automático
\end{topics}


%-----------------------------------------------------------------------------------
% NO BORRAR ESTAS LINEAS!
%-----------------------------------------------------------------------------------
\vspace{0.8cm}
]
%-----------------------------------------------------------------------------------


%===================================================================================

%===================================================================================
% Introducción
%-----------------------------------------------------------------------------------
\section{Introducción}\label{sec:intro}
%-----------------------------------------------------------------------------------
En los últimos tiempos ha habido una explosión en la investigación y aplicación del aprendizaje automático, en inglés \textit{machine learning} (ML)~\cite{hey2020machinelearning}. Sin embargo, el rendimiento de muchos métodos de aprendizaje automático es sensible a una gran variedad de decisiones~\cite{dyrmishi2019decision, radwa2019automated}, lo que constituye una barrera para nuevos usuarios~\cite{crisan2021fits}. Por ejemplo, el científico de datos debe seleccionar entre una amplia gama de posibles algoritmos, incluidas las técnicas de clasificación o regresión (como \textit{support vector machines}, redes neuronales, modelos bayesianos, árboles de decisión, etc.) y ajustar numerosos hiperparámetros del algoritmo seleccionado. Además, el rendimiento del modelo también se puede juzgar por varias métricas (por ejemplo, precisión, sensibilidad, medida F1). Incluso los expertos requieren gran cantidad de recursos y tiempo para crear modelos con buen rendimiento a causa del proceso de prueba y error que es repetido en cada aplicación para desarrollar modelos eficientes de aprendizaje automático.

Por estas razones ha emergido una nueva idea para automatizar el proceso de ML, aprendizaje de máquinas automático, denominada \textit{Automated Machine Learning} o AutoML. AutoML abarca el diseño de técnicas para automatizar y facilitar todo el proceso de implementación, experimentación y despliegue de algoritmos de aprendizaje automático. AutoML está concebido para reducir la carga de trabajo de los científicos de datos y permitir a los expertos construir automáticamente aplicaciones de ML sin mucho conocimiento en el campo. Por lo tanto, AutoML hace accesible enfoques de aprendizaje automático a los usuarios no expertos que están interesados en aplicarlos, pero no tienen los recursos para aprender sobre las tecnologías involucradas en detalle~\cite{hutter2019autmlbook}.

Sin embargo, una de las limitaciones presentes en los primeros sistemas de AutoML consiste en su inhabilidad de reusar conocimiento previo para solucionar nuevas tareas~\cite{dyrmishi2019decision}. Para cerrar esta brecha, las herramientas de AutoML comenzaron a aplicar técnicas de meta-learning, las cuales tienen el objetivo de obtener modelos para nuevas tareas usando experiencias previas. Meta-learning, o \textit{aprender a aprender}, es la ciencia de observar sistemáticamente cómo se desempeñan los diferentes enfoques de aprendizaje automático en una amplia gama de tareas de aprendizaje, y luego aprender de esta experiencia, o meta-datos, para aprender nuevas tareas mucho más rápido de lo que sería posible de otra manera. Esto no solo acelera y mejora drásticamente el diseño de algoritmos de aprendizaje automático, sino que también nos permite reemplazar algoritmos diseñados a mano con enfoques novedosos aprendidos de una manera basada en datos. Este tipo de estrategias ayudan a disminuir el costo de aplicar AutoML, al relacionar un nuevo conjunto de datos con los mejores flujos obtenidos en problemas similares previamente resueltos. 

En los recientes años se ha desarrollado un substancial interés en el campo de meta-learning y muchos sistemas de AutoML lo han integrado~\cite{fuerer2015efficient, maher2019smartml, drori2018alphad3m, yang2018oboe, zimmer2021auto, Feurer2020AutoSklearn2T}. Sin embargo, estas herramientas de meta-learning no son suficientemente flexibles para ser utilizadas en problemas prácticos que requieren la combinación de algoritmos y tecnologías de diferente naturaleza. Las técnicas actuales de meta-learning se centran principalmente en un subconjunto específico de algoritmos, a menudo adaptados a una biblioteca o conjunto de herramientas. Resolver problemas complejos, por otro lado, requiere la combinación de diferentes herramientas que podrían no estar disponibles en una misma biblioteca. Para la aplicación de meta-learning es necesario la representación de estos problemas mediante caracterizaciones informativas para los datasets y representaciones descriptivas para las soluciones obtenidas mediante diferentes herramientas. De esta forma, es posible que meta-learning sea capaz de resolver una gran cantidad de tareas.

El objetivo general de este trabajo es el diseño de una estrategia de meta-learning para métodos genéricos de AutoML, a partir de la combinación de técnicas de aprendizaje automático y optimización. La estrategia implementada tendrá el objetivo de acelerar el proceso de búsqueda de AutoML añadiendo conocimiento previo, de tal manera que se obtengan mejores resultados en el mismo período de tiempo.

 Dado un dataset, una tarea de evaluación (por ejemplo, clasificación o regresión), el algoritmo de meta-learning propuesto tiene el objetivo de producir una lista de los modelos candidatos, basada en el rendimiento esperado de estos modelos en el dataset dado. Esta lista es producida solamente con meta-conocimiento ganado del análisis de datasets relacionados y el entrenamiento de combinaciones de algoritmos en dichos datasets, sin ejecutar ninguno de los algoritmos candidatos. Teniendo este meta-conocimiento, es posible estimar el rendimiento de esos flujos y sugerirlos. Esta estimación, aunque no es exacta, mejorará el proceso de búsqueda de sistemas de aprendizaje de máquinas automático.

 El enfoque de meta-learning propuesto está compuesto por dos fases principales: la fase offline, de aprendizaje y la fase online, de recomendación. El objetivo de la fase offline es obtener los meta-datos necesarios para la solución del problema de meta-learning propuesto: la obtención de un ranking de modelos de aprendizaje para una determinada tarea. En esta fase se obtiene una caracterización de los datasets y el rendimiento y la estructura de un conjunto soluciones en dichos datasets. Por otro lado en la fase online, dada una tarea con los meta-datos ganados del análisis de las tareas similares y el entrenamiento de un conjunto de algoritmos en dichos datasets, el objetivo es producir una lista de las soluciones prometedoras para resolver la tarea inicial. Esta lista será utilizada para sugerir rápidamente algunas inicializaciones para el proceso de búsqueda de algoritmos de AutoGOAL.

%===================================================================================



%===================================================================================
% Desarrollo
%-----------------------------------------------------------------------------------
\section{Desarrollo}\label{sec:dev}
%-----------------------------------------------------------------------------------
  En esta sección (o secciones) incluya el contenido fundamental del artículo.
  No es necesario tener una sección nombrada \emph{Desarrollo}, por el contrario,
  nombre las secciones según el contenido que tratan.

%-----------------------------------------------------------------------------------
	\subsection{Organización del Documento}\label{sub:results}
%-----------------------------------------------------------------------------------
		Puede agregar secciones y subsecciones según sea necesario para organizar
		de manera más coherente su artículo. Tenga en cuenta que un documento más
		plano es más fácil de navegar y entender, pero las subsecciones relacionadas
		deberían estar agrupadas en una sección común.

		Los nombres de las secciones deben ir en mayúsculas, excepto para las
		preposiciones, conjunciones, y otros vocablos auxiliares.

		Empiece un nuevo párrafo cada vez que vaya a comenzar una idea nueva.

%-----------------------------------------------------------------------------------
	\subsection{Listas y Descripciones}\label{sub:lists}
%-----------------------------------------------------------------------------------
		Para producir listas enumeradas, utilice el siguiente estilo:
		\begin{enumerate}
			\item Primer Elemento
			\item Segundo Elemento
			%
			\begin {enumerate}
				\item {Segundo Elemento - Subítem Uno}
				\item {Segundo Elemento - Subítem Dos}
			\end {enumerate}
			%
		\end{enumerate}

%-----------------------------------------------------------------------------------
		Para producir descripciones, use el siguiente estilo:

%-----------------------------------------------------------------------------------
		\begin{description}
			\item [Primer Elemento] con su respectiva descripción.
			\item [Segundo Elemento] también con su respectiva descripción.
		\end{description}

%-----------------------------------------------------------------------------------
	\subsection{Figuras}\label{sub:figures}
%-----------------------------------------------------------------------------------
		Para producir cuerpos flotantes (figuras o tablas), asegúrese de numerar
		y etiquetar correctamente cada figura. Las referencias a las figuras deben
		estar correctamente etiquetadas. Por ejemplo, véase la Fig. \ref{fig:ex}\ldots

		\begin{figure}[h!]%
		\begin{center}
			\begin{tabular}{|c|c|c|} \hline
			 			& Método 1 	& Método 2 	\\ \hline
			A 			&  			&  			\\ \hline
			B			& 			& 			\\ \hline
			C 			& 			&  			\\ \hline
			\end{tabular}
		\caption{Figura de ejemplo. Recuerde especificar el origen de los datos que se muestran. \label{fig:ex}}
		\end{center}
		\end{figure}

%-----------------------------------------------------------------------------------
	\subsection{Código Fuente}\label{sub:listings}
%-----------------------------------------------------------------------------------
		Para producir código fuente, envuélvalo en una figura flotante y
		etiquételo correctamente. Por ejemplo, en la Fig. \ref{fig:code}
		se muestra un código bastante conocido\ldots

		% Configuración de Listings
		\lstset{keywordstyle=\color{blue}, basicstyle=\small}

		\begin{figure}[htb]%
			\begin{lstlisting}[language=c]%

    int main(int argc, char** argv)
    {
        // Imprimiendo "Hola Mundo".
        printf("Hello, World");
    }

			\end{lstlisting}
		\caption{Código fuente de ejemplo.\label{fig:code}}
		\end{figure}

%-----------------------------------------------------------------------------------
	\subsection{Referencias}
%-----------------------------------------------------------------------------------
  	Las referencias deben estar agrupadas en una sección al final del artículo,
  	y las citas numeradas correctamente, por ejemplo \cite{knuth} o \cite{goedel}.
  	Incluya toda la información importante de cada referencia, incluídos autor,
  	título, y notas de la edición. En caso de citar sitios web, además
  	de la URL, incluya la fecha en que fue consultado, como en \cite{wiki}. Numere 
  	las referencias según el orden en que se les cita.

%===================================================================================



%===================================================================================
% Conclusiones
%-----------------------------------------------------------------------------------
\section{Conclusiones}\label{sec:conc}

  En esta sección puede incluir las conclusiones de su investigación y las ideas
  sobre la continuidad del trabajo, en el caso que aplique.

%===================================================================================



%===================================================================================
% Recomendaciones
%-----------------------------------------------------------------------------------
\section{Recomendaciones}\label{sec:rec}

  En esta sección puede incluir recomendaciones sobre posibles formas de continuar
  la investigación u otros temas relacionados.

%===================================================================================



%===================================================================================
% Bibliografía
%-----------------------------------------------------------------------------------
\bibliographystyle{babplain-uh}
\bibliography{references}

%-----------------------------------------------------------------------------------

\label{end}

\end{document}

%===================================================================================
