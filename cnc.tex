%===================================================================================
% JORNADA CIENTÍFICA ESTUDIANTIL - MATCOM, UH
%===================================================================================
% Esta plantilla ha sido diseñada para ser usada en los artículos de la
% Jornada Científica Estudiantil de MatCom.
%
% Por favor, siga las instrucciones de esta plantilla y rellene en las secciones
% correspondientes.
%
% NOTA: Necesitará el archivo 'jcematcom.sty' en la misma carpeta donde esté este
%       archivo para poder utilizar esta plantila.
%===================================================================================



%===================================================================================
% PREÁMBULO
%-----------------------------------------------------------------------------------
\documentclass[a4paper,10pt,twocolumn]{article}

%===================================================================================
% Paquetes
%-----------------------------------------------------------------------------------
\usepackage{amsmath}
\usepackage{amsfonts}
\usepackage{amssymb}
\usepackage{jcematcom}
\usepackage[utf8]{inputenc}
\usepackage{listings}
\usepackage[pdftex]{hyperref}
\usepackage{caption}
\usepackage{subcaption}
%-----------------------------------------------------------------------------------
% Configuración
%-----------------------------------------------------------------------------------
\hypersetup{colorlinks,%
	    citecolor=black,%
	    filecolor=black,%
	    linkcolor=black,%
	    urlcolor=blue}

%===================================================================================



%===================================================================================
% Presentacion
%-----------------------------------------------------------------------------------
% Título
%-----------------------------------------------------------------------------------
\title{Una estrategia de Meta-Learning para flujos genéiricos de AutoML}

%-----------------------------------------------------------------------------------
% Autores
%-----------------------------------------------------------------------------------
\author{\\
\name Loraine Monteagudo García \email \href{loraine.monteagudo@matcom.uh.cu}{loraine.monteagudo@matcom.uh.cu}
	% \\ \addr Grupo B612 
	}

%-----------------------------------------------------------------------------------
% Tutores
%-----------------------------------------------------------------------------------
\tutors{\\
Dr. Suilan Estévez Velarde, \emph{Universidad de La Habana} \\
Lic. Daniel Alejandro Valdés Pérez, \emph{Universidad de La Habana}}

%-----------------------------------------------------------------------------------
% Headings
%-----------------------------------------------------------------------------------
\jcematcomheading{\the\year}{1-\pageref{end}}{L. Monteagudo}

%-----------------------------------------------------------------------------------
\ShortHeadings{Una estrategia de Meta-Learning}{L. Monteagudo}
%===================================================================================



%===================================================================================
% DOCUMENTO
%-----------------------------------------------------------------------------------
\begin{document}

%-----------------------------------------------------------------------------------
% NO BORRAR ESTA LINEA!
%-----------------------------------------------------------------------------------
\twocolumn[
%-----------------------------------------------------------------------------------

\maketitle

%===================================================================================
% Resumen y Abstract
%-----------------------------------------------------------------------------------
\selectlanguage{spanish} % Para producir el documento en Español

%-----------------------------------------------------------------------------------
% Resumen en Español
%-----------------------------------------------------------------------------------
\begin{abstract}

	El resumen en español debe constar de $100$ a $200$ palabras y presentar de forma
	clara y concisa el contenido fundamental del artículo.

\end{abstract}

%-----------------------------------------------------------------------------------
% English Abstract
%-----------------------------------------------------------------------------------
\vspace{0.5cm}

\begin{enabstract}

  The English abstract must have have $100$ to $200$ words, and present 
  the essentials of the article content in a clear and concise form.

\end{enabstract}

%-----------------------------------------------------------------------------------
% Palabras clave
%-----------------------------------------------------------------------------------
\begin{keywords}
	Separadas,
	Por,
	Comas.
\end{keywords}

%-----------------------------------------------------------------------------------
% Temas
%-----------------------------------------------------------------------------------
\begin{topics}
	Tema, Subtema.
\end{topics}


%-----------------------------------------------------------------------------------
% NO BORRAR ESTAS LINEAS!
%-----------------------------------------------------------------------------------
\vspace{0.8cm}
]
%-----------------------------------------------------------------------------------


%===================================================================================

%===================================================================================
% Introducción
%-----------------------------------------------------------------------------------
\section{Introducción}\label{sec:intro}
%-----------------------------------------------------------------------------------
  Este documento proporciona una plantilla para confeccionar el artículo de un 
  trabajo a presentar en la Jornada Científica Estudiantil. El artículo no excederá 
  las 7 páginas. En esta sección puede incluir una presentación del dominio de su 
  problema, los objetivos y motivaciones fundamentales de su investigación así como 
  un resumen del estado del arte al respecto.

%===================================================================================



%===================================================================================
% Desarrollo
%-----------------------------------------------------------------------------------
\section{Desarrollo}\label{sec:dev}
%-----------------------------------------------------------------------------------
  En esta sección (o secciones) incluya el contenido fundamental del artículo.
  No es necesario tener una sección nombrada \emph{Desarrollo}, por el contrario,
  nombre las secciones según el contenido que tratan.

%-----------------------------------------------------------------------------------
	\subsection{Organización del Documento}\label{sub:results}
%-----------------------------------------------------------------------------------
		Puede agregar secciones y subsecciones según sea necesario para organizar
		de manera más coherente su artículo. Tenga en cuenta que un documento más
		plano es más fácil de navegar y entender, pero las subsecciones relacionadas
		deberían estar agrupadas en una sección común.

		Los nombres de las secciones deben ir en mayúsculas, excepto para las
		preposiciones, conjunciones, y otros vocablos auxiliares.

		Empiece un nuevo párrafo cada vez que vaya a comenzar una idea nueva.

%-----------------------------------------------------------------------------------
	\subsection{Listas y Descripciones}\label{sub:lists}
%-----------------------------------------------------------------------------------
		Para producir listas enumeradas, utilice el siguiente estilo:
		\begin{enumerate}
			\item Primer Elemento
			\item Segundo Elemento
			%
			\begin {enumerate}
				\item {Segundo Elemento - Subítem Uno}
				\item {Segundo Elemento - Subítem Dos}
			\end {enumerate}
			%
		\end{enumerate}

%-----------------------------------------------------------------------------------
		Para producir descripciones, use el siguiente estilo:

%-----------------------------------------------------------------------------------
		\begin{description}
			\item [Primer Elemento] con su respectiva descripción.
			\item [Segundo Elemento] también con su respectiva descripción.
		\end{description}

%-----------------------------------------------------------------------------------
	\subsection{Figuras}\label{sub:figures}
%-----------------------------------------------------------------------------------
		Para producir cuerpos flotantes (figuras o tablas), asegúrese de numerar
		y etiquetar correctamente cada figura. Las referencias a las figuras deben
		estar correctamente etiquetadas. Por ejemplo, véase la Fig. \ref{fig:ex}\ldots

		\begin{figure}[h!]%
		\begin{center}
			\begin{tabular}{|c|c|c|} \hline
			 			& Método 1 	& Método 2 	\\ \hline
			A 			&  			&  			\\ \hline
			B			& 			& 			\\ \hline
			C 			& 			&  			\\ \hline
			\end{tabular}
		\caption{Figura de ejemplo. Recuerde especificar el origen de los datos que se muestran. \label{fig:ex}}
		\end{center}
		\end{figure}

%-----------------------------------------------------------------------------------
	\subsection{Código Fuente}\label{sub:listings}
%-----------------------------------------------------------------------------------
		Para producir código fuente, envuélvalo en una figura flotante y
		etiquételo correctamente. Por ejemplo, en la Fig. \ref{fig:code}
		se muestra un código bastante conocido\ldots

		% Configuración de Listings
		\lstset{keywordstyle=\color{blue}, basicstyle=\small}

		\begin{figure}[htb]%
			\begin{lstlisting}[language=c]%

    int main(int argc, char** argv)
    {
        // Imprimiendo "Hola Mundo".
        printf("Hello, World");
    }

			\end{lstlisting}
		\caption{Código fuente de ejemplo.\label{fig:code}}
		\end{figure}

%-----------------------------------------------------------------------------------
	\subsection{Referencias}
%-----------------------------------------------------------------------------------
  	Las referencias deben estar agrupadas en una sección al final del artículo,
  	y las citas numeradas correctamente, por ejemplo \cite{knuth} o \cite{goedel}.
  	Incluya toda la información importante de cada referencia, incluídos autor,
  	título, y notas de la edición. En caso de citar sitios web, además
  	de la URL, incluya la fecha en que fue consultado, como en \cite{wiki}. Numere 
  	las referencias según el orden en que se les cita.

%===================================================================================



%===================================================================================
% Conclusiones
%-----------------------------------------------------------------------------------
\section{Conclusiones}\label{sec:conc}

  En esta sección puede incluir las conclusiones de su investigación y las ideas
  sobre la continuidad del trabajo, en el caso que aplique.

%===================================================================================



%===================================================================================
% Recomendaciones
%-----------------------------------------------------------------------------------
\section{Recomendaciones}\label{sec:rec}

  En esta sección puede incluir recomendaciones sobre posibles formas de continuar
  la investigación u otros temas relacionados.

%===================================================================================



%===================================================================================
% Bibliografía
%-----------------------------------------------------------------------------------
\begin{thebibliography}{99}
%-----------------------------------------------------------------------------------
	\bibitem{knuth} Donald E. Knuth. \emph{The Art of Computer Programming}.
		Volume 1: Fundamental Algorithms (3rd~edition), 1997.
		Addison-Wesley Professional.

	\bibitem{goedel} Kurt Göedel. \emph{Über formal unentscheidbare Sätze der
		Principia Mathematica und verwandter Systeme, I}.
		Monatshefte für Mathematik und Physik 38.

	\bibitem{wiki} Wikipedia. URL: \href{http://en.wikipedia.org}
	  {http://en.wikipedia.org}.
		Consultado en \today.

%-----------------------------------------------------------------------------------
\end{thebibliography}

%-----------------------------------------------------------------------------------

\label{end}

\end{document}

%===================================================================================
