%===================================================================================
% JORNADA CIENTÍFICA ESTUDIANTIL - MATCOM, UH
%===================================================================================
% Esta plantilla ha sido diseñada para ser usada en los artículos de la
% Jornada Científica Estudiantil de MatCom.
%
% Por favor, siga las instrucciones de esta plantilla y rellene en las secciones
% correspondientes.
%
% NOTA: Necesitará el archivo 'jcematcom.sty' en la misma carpeta donde esté este
%       archivo para poder utilizar esta plantila.
%===================================================================================



%===================================================================================
% PREÁMBULO
%-----------------------------------------------------------------------------------
\documentclass[a4paper,10pt,twocolumn]{article}

%===================================================================================
% Paquetes
%-----------------------------------------------------------------------------------
\usepackage{amsmath}
\usepackage{amsfonts}
\usepackage{amssymb}
\usepackage{jcematcom}
\usepackage[utf8]{inputenc}
\usepackage{listings}
\usepackage[pdftex]{hyperref}
\usepackage{caption}
\usepackage{subcaption}
%-----------------------------------------------------------------------------------
% Configuración
%-----------------------------------------------------------------------------------
\hypersetup{colorlinks,%
	    citecolor=black,%
	    filecolor=black,%
	    linkcolor=black,%
	    urlcolor=blue}

%===================================================================================



%===================================================================================
% Presentacion
%-----------------------------------------------------------------------------------
% Título
%-----------------------------------------------------------------------------------
\title{Una estrategia de Meta-Learning para flujos genéricos de AutoML}

%-----------------------------------------------------------------------------------
% Autores
%-----------------------------------------------------------------------------------
\author{\\
\name Loraine Monteagudo García \email \href{loraine.monteagudo@matcom.uh.cu}{loraine.monteagudo@matcom.uh.cu}
	% \\ \addr Grupo B612 
	}

%-----------------------------------------------------------------------------------
% Tutores
%-----------------------------------------------------------------------------------
\tutors{\\
Dr. Suilan Estévez Velarde, \emph{Universidad de La Habana} \\
Lic. Daniel Alejandro Valdés Pérez, \emph{Universidad de La Habana}}

%-----------------------------------------------------------------------------------
% Headings
%-----------------------------------------------------------------------------------
\jcematcomheading{\the\year}{1-\pageref{end}}{L. Monteagudo}

%-----------------------------------------------------------------------------------
\ShortHeadings{Una estrategia de Meta-Learning}{L. Monteagudo}
%===================================================================================



%===================================================================================
% DOCUMENTO
%-----------------------------------------------------------------------------------
\begin{document}

%-----------------------------------------------------------------------------------
% NO BORRAR ESTA LINEA!
%-----------------------------------------------------------------------------------
\twocolumn[
%-----------------------------------------------------------------------------------

\maketitle

%===================================================================================
% Resumen y Abstract
%-----------------------------------------------------------------------------------
\selectlanguage{spanish} % Para producir el documento en Español

%-----------------------------------------------------------------------------------
% Resumen en Español
%-----------------------------------------------------------------------------------
\begin{abstract}

	El campo de aprendizaje de máquinas automático (AutoML) se ha destacado como una de las principales alternativas para encontrar buenas soluciones para problemas complejos de aprendizaje automático. A pesar del reciente éxito de AutoML, todavía quedan muchos desafíos. El aprendizaje de AutoML es un proceso costoso en tiempo y puede llegar a ser ineficiente computacionalmente. Meta-Learning es descrito como el proceso de aprender de experiencias pasadas aplicando varios algoritmos de aprendizaje en diferentes tipos de datos y, por lo tanto, reduce el tiempo necesario para aprender nuevas tareas. Una de las ventajas de las técnicas de meta-learning es que pueden servir como un apoyo eficiente para el proceso de AutoML, aprendiendo de tareas previas los mejores algoritmos para resolver un determinado tipo de problema. De esta manera, es posible acelerar el proceso de AutoML, obteniendo mejores resultados en el mismo período de tiempo. El objetivo de esta tesis es diseñar una estrategia de meta-learning para dominios genéricos en el aprendizaje automático.

\end{abstract}

%-----------------------------------------------------------------------------------
% English Abstract
%-----------------------------------------------------------------------------------
\vspace{0.5cm}

\begin{enabstract}
	The field of automated machine learning (AutoML) has been highlighted as one of the main alternatives for finding good solutions for complex machine learning problems. Despite the recent success of AutoML, many challenges remain. Learning AutoML is a time-consuming process and can be computationally inefficient. Meta-learning is described as the process of learning from past experiences by applying various learning algorithms on different types of data, and therefore reduces the time required to learn new tasks. One of the advantages of meta-learning techniques is that they can serve as an efficient support for the AutoML process, learning from previous tasks the best algorithms to solve a certain type of problem. In this way, it is possible to speed up the AutoML process, obtaining better results in the same period of time. The objective of this thesis is to design a meta-learning strategy for generic domains in machine learning.

\end{enabstract}

%-----------------------------------------------------------------------------------
% Palabras clave
%-----------------------------------------------------------------------------------
\begin{keywords}
	Aprendizaje Automático, Meta-Learning, AutoML
\end{keywords}

%-----------------------------------------------------------------------------------
% Temas
%-----------------------------------------------------------------------------------
\begin{topics}
	Inteligencia Artificial, Aprendizaje Automático
\end{topics}


%-----------------------------------------------------------------------------------
% NO BORRAR ESTAS LINEAS!
%-----------------------------------------------------------------------------------
\vspace{0.8cm}
]
%-----------------------------------------------------------------------------------


%===================================================================================

%===================================================================================
% Introducción
%-----------------------------------------------------------------------------------
\section{Introducción}\label{sec:intro}
%-----------------------------------------------------------------------------------
En los últimos tiempos ha habido una explosión en la investigación y aplicación del aprendizaje automático, en inglés \textit{machine learning} (ML)~\cite{hey2020machinelearning}. Sin embargo, el rendimiento de muchos métodos de aprendizaje automático es sensible a una gran variedad de decisiones~\cite{dyrmishi2019decision, radwa2019automated}, lo que constituye una barrera para nuevos usuarios~\cite{crisan2021fits}. Por ejemplo, el científico de datos debe seleccionar entre una amplia gama de posibles algoritmos, incluidas las técnicas de clasificación o regresión (como \textit{support vector machines}, redes neuronales, modelos bayesianos, árboles de decisión, etc.) y ajustar numerosos hiperparámetros del algoritmo seleccionado. Además, el rendimiento del modelo también se puede juzgar por varias métricas (por ejemplo, precisión, sensibilidad, medida F1). Incluso los expertos requieren gran cantidad de recursos y tiempo para crear modelos con buen rendimiento a causa del proceso de prueba y error que es repetido en cada aplicación para desarrollar modelos eficientes de aprendizaje automático.

Por estas razones ha emergido una nueva idea para automatizar el proceso de ML, aprendizaje de máquinas automático, denominada \textit{Automated Machine Learning} o AutoML. AutoML abarca el diseño de técnicas para automatizar y facilitar todo el proceso de implementación, experimentación y despliegue de algoritmos de aprendizaje automático. AutoML está concebido para reducir la carga de trabajo de los científicos de datos y permitir a los expertos construir automáticamente aplicaciones de ML sin mucho conocimiento en el campo. Por lo tanto, AutoML hace accesible enfoques de aprendizaje automático a los usuarios no expertos que están interesados en aplicarlos, pero no tienen los recursos para aprender sobre las tecnologías involucradas en detalle~\cite{hutter2019autmlbook}.

Sin embargo, una de las limitaciones presentes en los primeros sistemas de AutoML consiste en su inhabilidad de reusar conocimiento previo para solucionar nuevas tareas~\cite{dyrmishi2019decision}. Para cerrar esta brecha, las herramientas de AutoML comenzaron a aplicar técnicas de meta-learning, las cuales tienen el objetivo de obtener modelos para nuevas tareas usando experiencias previas. Meta-learning, o \textit{aprender a aprender}, es la ciencia de observar sistemáticamente cómo se desempeñan los diferentes enfoques de aprendizaje automático en una amplia gama de tareas de aprendizaje, y luego aprender de esta experiencia, o meta-datos, para aprender nuevas tareas mucho más rápido de lo que sería posible de otra manera. Esto no solo acelera y mejora drásticamente el diseño de algoritmos de aprendizaje automático, sino que también nos permite reemplazar algoritmos diseñados a mano con enfoques novedosos aprendidos de una manera basada en datos. Este tipo de estrategias ayudan a disminuir el costo de aplicar AutoML, al relacionar un nuevo conjunto de datos con los mejores flujos obtenidos en problemas similares previamente resueltos. 

En los recientes años se ha desarrollado un substancial interés en el campo de meta-learning y muchos sistemas de AutoML lo han integrado~\cite{fuerer2015efficient, maher2019smartml, drori2018alphad3m, yang2018oboe, zimmer2021auto, Feurer2020AutoSklearn2T}. Sin embargo, estas herramientas de meta-learning no son suficientemente flexibles para ser utilizadas en problemas prácticos que requieren la combinación de algoritmos y tecnologías de diferente naturaleza. Las técnicas actuales de meta-learning se centran principalmente en un subconjunto específico de algoritmos, a menudo adaptados a una biblioteca o conjunto de herramientas. Resolver problemas complejos, por otro lado, requiere la combinación de diferentes herramientas que podrían no estar disponibles en una misma biblioteca. Para la aplicación de meta-learning es necesario la representación de estos problemas mediante caracterizaciones informativas para los datasets y representaciones descriptivas para las soluciones obtenidas mediante diferentes herramientas. De esta forma, es posible que meta-learning sea capaz de resolver una gran cantidad de tareas.

El objetivo general de este trabajo es el diseño de una estrategia de meta-learning para métodos genéricos de AutoML, a partir de la combinación de técnicas de aprendizaje automático y optimización. La estrategia implementada tendrá el objetivo de acelerar el proceso de búsqueda de AutoML añadiendo conocimiento previo, de tal manera que se obtengan mejores resultados en el mismo período de tiempo.

 Dado un dataset, una tarea de evaluación (por ejemplo, clasificación o regresión), el algoritmo de meta-learning propuesto tiene el objetivo de producir una lista de los modelos candidatos, basada en el rendimiento esperado de estos modelos en el dataset dado. Esta lista es producida solamente con meta-conocimiento ganado del análisis de datasets relacionados y el entrenamiento de combinaciones de algoritmos en dichos datasets, sin ejecutar ninguno de los algoritmos candidatos. Teniendo este meta-conocimiento, es posible estimar el rendimiento de esos flujos y sugerirlos. Esta estimación, aunque no es exacta, mejorará el proceso de búsqueda de sistemas de aprendizaje de máquinas automático.

 El enfoque de meta-learning propuesto está compuesto por dos fases principales: la fase offline, de aprendizaje y la fase online, de recomendación. El objetivo de la fase offline es obtener los meta-datos necesarios para la solución del problema de meta-learning propuesto: la obtención de un ranking de modelos de aprendizaje para una determinada tarea. En esta fase se obtiene una caracterización de los datasets y el rendimiento y la estructura de un conjunto soluciones en dichos datasets. Por otro lado en la fase online, dada una tarea con los meta-datos ganados del análisis de las tareas similares y el entrenamiento de un conjunto de algoritmos en dichos datasets, el objetivo es producir una lista de las soluciones prometedoras para resolver la tarea inicial. Esta lista será utilizada para sugerir rápidamente algunas inicializaciones para el proceso de búsqueda de algoritmos de AutoGOAL.

%===================================================================================



%===================================================================================
% Estado del Arte
%-----------------------------------------------------------------------------------
\section{Estado del Arte}\label{sec:review}
%-----------------------------------------------------------------------------------
En esta seccion se proporciona una introducción a los campos y los trabajos que están relacionados con las técnicas utilizadas en este trabajo. Se comienza introduciendo las ideas básicas de meta-learning (\ref{sub:metalearning}), definiendo el problema que este campo resuelve, explicando la estructura de un sistema de meta-learning y varias de sus aplicaciones. El objetivo fundamental de este trabajo es añadir componentes de meta-learning a un sistema de \textit{Automated Machine Learning} (AutoML), así que este campo es introducido (\ref{sub:automl}). Se presentan diferentes formulaciones teóricas del problema de AutoML y se describen los componentes fundamentales de un proceso de AutoML, así como ejemplos de sistemas para cada uno de los enfoques existentes. El área de interés de esta investigación es la aplicación de meta-learning para la selección de modelos, en concreto, su utilización para añadir conocimiento en sistemas AutoML, por lo que varias técnicas para la solución de este problema son estudiadas (\ref{sub:metalearning-automl}).
%-----------------------------------------------------------------------------------
	\subsection{Meta-Learning}\label{sub:metalearning}
%-----------------------------------------------------------------------------------
Meta-learning es mejor entendido comúnmente como ``aprendiendo a aprender'', lo cual se refiere al proceso de mejorar un algoritmo de aprendizaje a través de múltiples episodios de aprendizaje. En contraste, el aprendizaje automático convencional mejora las predicciones del modelo sobre múltiples instancias de datos. Durante el \textit{base-learning} o aprendizaje base, un algoritmo de aprendizaje interior (o inferior/base) resuelve una tarea como clasificación de imágenes, definida por un dataset y un objetivo. Durante \emph{meta-learning}, un algoritmo externo (o superior/meta) actualiza el algoritmo interior de tal manera que el modelo que aprende mejora un objetivo externo. Los episodios de aprendizaje de la tarea base pueden ser vistos como una forma de proveer las instancias necesitadas por el algoritmo externo para aprender el algoritmo de aprendizaje base~\cite{hospedales2021metalearning}. 

Meta-learning difiere de \textit{base-learning} en el alcance del nivel de adaptación. Mientras que el aprendizaje en un nivel base está enfocado en acumular experiencia en una tarea específica, el aprendizaje en meta-learning tiene el objetivo de acumular experiencia en el rendimiento de múltiples aplicaciones de un sistema de aprendizaje. De esta forma, muchos algoritmos convencionales tales como la búsqueda aleatoria de hiperparámetros mediante validación cruzada podrían caer en la definición de meta-learning. La característica destacada del \emph{meta-learning} contemporáneo es un meta-objetivo explícitamente definido, y una optimización de extremo a extremo del algoritmo interior con respecto a este objetivo.

Un sistema de meta-learning está compuesto esencialmente por dos partes. Una parte tiene la tarea de adquirir meta-conocimiento de sistemas de aprendizaje automático. La otra parte tiene el objetivo de aplicar este meta-conocimiento a nuevos problemas con el objetivo de identificar un algoritmo o técnica de aprendizaje óptimo~\cite{bradzil2017metalearning}.

Meta-learning puede ser empleada en una variedad de configuraciones, con cierto desacuerdo en la literatura sobre lo que constituye exactamente un problema de meta-learning. Meta-learning es extremadamente útil en los casos donde es requerido un modelo de aprendizaje automático y hay poca cantidad de datos, ya que el modelo contiene muchos parámetros que no pueden ser estimados precisamente con pocos datos. Algunas de las aplicaciones comunes son en la investigación robótica, donde se espera que los robots tengan un mayor nivel de autonomía en IA, en el descubrimiento de drogas para manejar los datos de altas dimensiones con un tamaño de muestra pequeño y en la traducción de lenguajes raramente usados~\cite{peng2020comprehensive}. Además, meta-learning es ampliamente empleado en el problema de selección de algoritmos, sobre esta aplicación se profundiza en la Sección \ref{sub:metalearning-automl}.

Meta-learning constituye una solución factible para los problemas donde una definición específica de ``tarea'' y ``etiqueta'' puede ser claramente distinguida. Un sistema de meta-learning es flexible y puede ser integrado convenientemente con la mayoría de los algoritmos de aprendizaje automático para proporcionar soluciones factibles~\cite{peng2020comprehensive}. Para las tareas que son computacionalmente costosas, meta-learning presenta la opción de agregación o adaptación de los resultados anteriores para salvar recursos computacionales.
 

%-----------------------------------------------------------------------------------
	\subsection{AutoML}\label{sub:automl}
%-----------------------------------------------------------------------------------

\textit{Automated Machine Learning} (AutoML) o Aprendizaje de Máquinas Automático es el campo que se enfoca en los métodos que tienen el objetivo de automatizar diferentes etapas del proceso de aprendizaje automático. Como su nombre indica, AutoML es la intersección de dos campos: automatización y ML. Las soluciones de AutoML están recibiendo incrementalmente más atención tanto por la comunidad de ML como por los usuarios por las grandes cantidades de datos disponibles en todas partes y la falta de expertos de aprendizaje automático que puedan supervisar/asesorar el desarrollo de sistemas basados en ML~\cite{hutter2019autmlbook}.

La comunidad de AutoML se ha centrado en resolver varias partes de un flujo de trabajo de aprendizaje automático estándar. Algunos ejemplos de estas partes o subtareas que son aplicadas en AutoML son:

\begin{itemize}
	\item Preparación Automática de Datos o \textit{Automated Data Preparation}
	
	\item Ingeniería Automática de Características o \textit{Automated Feature Engineering}
	
	\item Búsqueda de Arquitecturas Neuronales o \textit{Neural Architecture Search} (NAS)
\end{itemize}

Sin embargo, los estudios recientes de AutoML buscan automatizar el flujo de algoritmos de aprendizaje automático entero~\cite{fuerer2015efficient, olson2019tpot, paszke2019pytorch, chen2018autostacker, swearingen2017atm}. Un flujo de algoritmos es una forma de codificar y automatizar el flujo de trabajo necesario para producir un modelo de aprendizaje automático. Los flujos de algoritmos de aprendizaje automático constan de varios pasos secuenciales que realizan desde la extracción de datos y el preprocesamiento hasta el entrenamiento y la implementación del modelo~\cite{web-mlpipe}.

Dos problemas importantes en AutoML son que ningún algoritmo de ML obtiene los mejores resultados en todos los datasets, también conocido como \textit{No Free Lunch Problem} \cite{wolpert1995no}, y que algunos métodos de aprendizaje automático dependen crucialmente de la optimización de hiperparámetros. Para la resolución de estos problemas AutoML se apoya de dos áreas o subtareas que constituyen su base: la selección de modelos (\textit{Model Selection}, MS)~\cite{thornton2013auto} y la optimización de hiperparámetros (\textit{Hyperparameter Optimization}, HPO)~\cite{fuerer2019hyperparameter}. La combinación de estas áreas se refiere al problema de AutoML como un problema de selección combinada de modelos y optimización de hiperparámetros (\textit{Combined Algorithm Selection and Hyperparameter Optimization}, CASH)~\cite{thornton2013auto}.

Con el objetivo de abordar el problema de CASH el proceso de AutoML consta de tres componentes que definen el proceso de optimización:

\begin{description}
	\item[Espacio de Búsqueda:] precisa los algoritmos y todos los rangos válidos para sus hiperparámetros que son posibles soluciones para un problema de AutoML concreto. Además, se pueden optimizar combinaciones complejas de algoritmos, en cuyo caso las restricciones de compatibilidad entre algoritmos también son modeladas.
	\item[Estrategia de Búsqueda:] detalla como se explora el espacio de búsqueda, que puede ser de tamaño exponencial o ilimitado. Se ve afectado por el clásico problema de Exploración vs. Explotación, ya que se quieren encontrar soluciones de alto rendimiento rápidamente, pero se debe evitar converger prematuramente a regiones subóptimas de búsqueda.
	\item[Estrategias de Estimación de Rendimiento:] son mecanismos para estimar la capacidad predictiva de las soluciones encontradas por los sistemas de AutoML.
\end{description}

La estrategia de búsqueda es el proceso que sustituye la búsqueda de los hiperparámetros realizada por los humanos. Este procedimiento requiere tiempo y recursos considerables debido a los métodos de prueba y error que son necesitados para buscar el mejor modelo y su configuración de hiperparámetros. Por lo tanto, muchos métodos de optimización han surgido con el objetivo de acelerar esta búsqueda para liberar a los humanos de este tedioso proceso y para explorar el espacio de búsqueda definido de forma automática. Este proceso de optimización es el que pretende imitar el rol de los expertos y es el núcleo fundamental para resolver el problema de CASH. Algunos ejemplos de las estrategias de búsquedas más usadas son:

\begin{description}
	\item[Grid Search y Random Search:] \textit{Grid Search} (GS) es el proceso de discretizar cada hiperparámetro y evaluar exhaustivamente cada combinación de valores. Por otro lado, \textit{Random Search} (RS) o búsqueda aleatoria configura una cuadrícula de valores de hiperparámetros y selecciona combinaciones aleatorias para entrenar el modelo. Esto permite controlar explícitamente el número de combinaciones de parámetros que se intentan. Algunos sistemas de AutoML que han implementado versiones de estas estrategias son: Hyperopt~\cite{bergstra2013hyperopt}, Rafiki~\cite{wang2018rafiki} y FLAML~\cite{wang2021flaml}.
 	\item[Optimización Bayesiana:] BO es un algoritmo iterativo, cuya idea clave es modelar la asignación entre un conjunto de hiperparámetros $\lambda$ y da como resultado una estimación de su rendimiento $\hat{c}(\lambda)$ basado en valores de rendimiento observados encontrados en un archivo $A$ mediante regresión no lineal. Este modelo aproximado se denomina modelo sustituto, o modelo probabilístico, para el cual es normalmente utilizado un proceso gausiano o un bosque aleatorio. Esta estrategia de búsqueda ha sido utilizada por los siguientes sistenmas: Auto-WEKA~\cite{thornton2013auto}, Hyperopt~\cite{bergstra2013hyperopt}, Auto-Sklearn~\cite{fuerer2015efficient}, Auto-Net~\cite{mendoza2016towards} y Auto-Keras~\cite{jin2019auto}
  	\item[Algoritmos Evolutivos:] Un algoritmo evolutivo (EA) es un subconjunto de la computación evolutiva, un algoritmo genérico de optimización metaheurística basado en la población. En un algoritmo evolutivo, una \textit{población} de soluciones candidatas (llamadas individuos, criaturas o fenotipos) en un problema de optimización evoluciona hacia mejores soluciones. Cada solución candidata tiene un conjunto de propiedades (sus cromosomas o genotipo) que se pueden mutar y alterar. En la terminología de la optimización de hiperparámetros un \textit{individuo} es una configuración de hiperparámetros única, la \textit{población} es un conjunto de configuraciones de hiperparámetros actualmente mantenido y la \textit{aptitud} de un individuo es su error de generalización. La mutación es el cambio (aleatorio) de uno o unos pocos valores de hiperparámetros en una configuración. El cruce crea una nueva configuración de hiperparámetros mezclando aleatoriamente los valores de otras dos configuraciones. Ejemplos del uso de algoritmos evolutivos el los sistemas de AutoML son: Autostacker~\cite{chen2018autostacker}, TPOT~\cite{olson2019tpot}, RECIPE~\cite{de2017recipe}, $Auto-MEKA_{GPP}$~\cite{de2018automated} y AutoGOAL~\cite{autogoal}.
   \item[Aprendizaje por refuerzo:] El aprendizaje por refuerzo (RL), como estrategia de búsqueda, consiste en entrenar un agente que realiza modificaciones sobre una solución con el objetivo de maximizar una recompensa que depende del rendimiento de dicha solución. Es un marco de optimización muy general y sólido, que puede resolver problemas con retroalimentación retardada. A diferencia de los métodos anteriores, las retroalimentaciones (es decir, la recompensa y el estado) no necesitan ser devueltos inmediatamente una vez que se toma una acción. Se pueden devolver después de realizar una secuencia de acciones. Un ejemplo del uso de esta estrategia se encuentra en Alpha3DM~\cite{drori2018alphad3m}
   \item[Monte Carlo Tree Search (MCTS):] Monte Carlo Tree Search (MCTS) es un algoritmo de búsqueda heurística para algunos tipos de procesos de decisión. El enfoque de MCTS es en el análisis de los movimientos más prometedores, expandiendo el árbol de búsqueda basado en un muestreo aleatorio del espacio de búsqueda. Cada ciclo de evaluación consiste en construir una solución completa, que se traduce en descender por una rama del árbol del espacio de búsqueda. En un algoritmo de Monte Carlo es necesario definir cómo se escoge el siguiente nodo a evaluar, lo que conlleva un balance entre exploración y explotación. A medida que se explora el espacio de búsqueda, se descubre qué decisiones en niveles superiores tienen un mejor rendimiento y se sesga la búsqueda hacia esas regiones del espacio. Ejemplos de la implementación de esta estrategia se encuentra en: MOSAIC~\cite{rakotoarison2019automated} 
\end{description}


%-----------------------------------------------------------------------------------
	\subsection{Meta-Learning para AutoML}\label{sub:metalearning-automl}
%-----------------------------------------------------------------------------------
		Para producir cuerpos flotantes (figuras o tablas), asegúrese de numerar
		y etiquetar correctamente cada figura. Las referencias a las figuras deben
		estar correctamente etiquetadas. Por ejemplo, véase la Fig. \ref{fig:ex}\ldots

		\begin{figure}[h!]%
		\begin{center}
			\begin{tabular}{|c|c|c|} \hline
			 			& Método 1 	& Método 2 	\\ \hline
			A 			&  			&  			\\ \hline
			B			& 			& 			\\ \hline
			C 			& 			&  			\\ \hline
			\end{tabular}
		\caption{Figura de ejemplo. Recuerde especificar el origen de los datos que se muestran. \label{fig:ex}}
		\end{center}
		\end{figure}

%-----------------------------------------------------------------------------------
	\subsection{Código Fuente}\label{sub:listings}
%-----------------------------------------------------------------------------------
		Para producir código fuente, envuélvalo en una figura flotante y
		etiquételo correctamente. Por ejemplo, en la Fig. \ref{fig:code}
		se muestra un código bastante conocido\ldots

		% Configuración de Listings
		\lstset{keywordstyle=\color{blue}, basicstyle=\small}

		\begin{figure}[htb]%
			\begin{lstlisting}[language=c]%

    int main(int argc, char** argv)
    {
        // Imprimiendo "Hola Mundo".
        printf("Hello, World");
    }

			\end{lstlisting}
		\caption{Código fuente de ejemplo.\label{fig:code}}
		\end{figure}

%-----------------------------------------------------------------------------------
	\subsection{Referencias}
%-----------------------------------------------------------------------------------
  	Las referencias deben estar agrupadas en una sección al final del artículo,
  	y las citas numeradas correctamente, por ejemplo \cite{knuth} o \cite{goedel}.
  	Incluya toda la información importante de cada referencia, incluídos autor,
  	título, y notas de la edición. En caso de citar sitios web, además
  	de la URL, incluya la fecha en que fue consultado, como en \cite{wiki}. Numere 
  	las referencias según el orden en que se les cita.

%===================================================================================



%===================================================================================
% Conclusiones
%-----------------------------------------------------------------------------------
\section{Conclusiones}\label{sec:conc}

  En esta sección puede incluir las conclusiones de su investigación y las ideas
  sobre la continuidad del trabajo, en el caso que aplique.

%===================================================================================



%===================================================================================
% Recomendaciones
%-----------------------------------------------------------------------------------
\section{Recomendaciones}\label{sec:rec}

  En esta sección puede incluir recomendaciones sobre posibles formas de continuar
  la investigación u otros temas relacionados.

%===================================================================================



%===================================================================================
% Bibliografía
%-----------------------------------------------------------------------------------
\bibliographystyle{babplain-uh}
\bibliography{references}

%-----------------------------------------------------------------------------------

\label{end}

\end{document}

%===================================================================================
